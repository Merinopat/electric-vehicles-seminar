Resale values for electric vehicles (EVs) are generally lower than for traditional petrol and diesel cars, primarily due to faster depreciation rates. Despite this, operational costs for EVs tend to be significantly lower, as they require less maintenance and the cost of electricity is usually cheaper than fuel.

A recent study examining EV pricing in Germany between 2015 and 2020 found that luxury and mid-sized EVs are expected to reach purchase price parity with conventional vehicles by 2023 and 2026, respectively, while small EVs may not do so before 2030. The total cost of ownership for larger EVs was already comparable to combustion vehicles by 2020, even without subsidies. However, small EVs still depend heavily on financial support, underlining the need for further innovation and policy incentives. \cite{goetzel2022empirical}

The internal combustion engine (ICE) systems in EREVs are less costly than in conventional ICE vehicles due to simpler transmission design. Electric motor costs are also declining steadily with increased production volumes. For EVs, additional expenses such as potential battery replacements—estimated based on load cycles—must be considered. While HEVs and FCHEVs generally avoid this due to limited battery use, BEVs and EREVs may require replacements over their lifetime. Other ongoing costs include maintenance, insurance, and vehicle taxes. A detailed total cost of ownership analysis in Germany, covering six vehicle classes, three user types, and various drive technologies for 2015, 2030, and 2050, found that only mild and full hybrids are economically viable without subsidies, primarily for high-mileage users. In contrast, fully electric and extended-range vehicles remain financially unfeasible without government support, despite policy focus on their adoption. \cite{owid-co2-emissions-from-transport}